\documentclass[oneside]{book}

\input{common}


\SetOrganization{Primordial Machine's}
\SetLibraryName{Vectors Library}

\SetLibraryIncludeFileName{include.hpp}
\SetLibraryIncludesDirectoryPath{primordialmachine/vectors/\newline\$(PlatformTarget.toLower())/\$(Configuration.toLower())/includes}

\SetLibraryIncludeDirectiveFilePath{primordialmachine/vectors/include.hpp}

\SetLibraryStaticLibrariesDirectoryPath{primordialmachine/vectors/\newline\$(PlatformTarget.toLower())/\$(Configuration.toLower())/libraries}
\SetLibraryStaticLibraryFileName{vectors.lib}


\SetLibraryVersion{v1.0}
\SetLibraryRepository{https://github.com/primordialmachine/vectors}
\SetAuthor{Michael Heilmann}
\SetEmail{michaelheilmann@primordialmachine.com}

\begin{document}
\maketitle
\tableofcontents
\chapter{Synopsis}
C++ 17 library providing floating point vectors over $n$ dimensions.
The library is made available publicly on
\href{\GetLibraryRepository}{Github}
under the
\href{\GetLibraryRepository/blob/master/LICENSE}{MIT License}.

\chapter{Limitations and Restrictions}
The library officially only supports Visual Studio 2017 and Windows 10.

\chapter{Introductory example}
\textit{\color{orange}This library does not provide any examples yet.}
%Examples are located in the \href{\GetLibraryRepository/blob/master/examples}{examples} directory.

% Copyright (c) 2018 Michael Heilmann. All rights reserved.
\chapter{Building under Visual Studio 2017}
\begin{enumerate}
\item Open the solution \texttt{solution.sln} in Microsoft Visual Studio 2017.
\item Batch build everything.
\item The folder \texttt{packages} contains the distribution of the library i.e. include files and the
      static libraries for
  \begin{enumerate}
    \item the platform targets \texttt{x86} and \texttt{x64} and
    \item configurations \texttt{Release} and \texttt{Debug}.
  \end{enumerate}
\item Copy the contents of the \verb+packages+ folder into a directory. Let
      \verb+[library home]+ be a placeholder denoting the path by which that folder
      can be referenced from your project.
\item Add
  \begin{enumerate}
    \item the include path
\texttt{[library home]/\GetLibraryIncludesDirectoryPath}
	and
    \item the library path
\texttt{[library home]/\GetLibraryStaticLibrariesDirectoryPath}
    to your project.
\end{enumerate}
\item Link your project with the library \texttt{\GetLibraryStaticLibraryFileName}.
\item Add the include directive \texttt{\#include "{}\GetLibraryIncludeDirectiveFilePath"{}} where appropriate.
\item You can now use the functionality provided by the library.
\end{enumerate}


\chapter{Library Interface Documentation}

\section{\texttt{namespace primordialmachine}}
The namespace this library is adding its declarations/definitions to.
The added namespace elements are documented below.

\section{\textit{concept $n,m$-ary-Functor}}
A \textit{$n,m$-ary-Functor} is a \textit{Functor} which provides an constant \texttt{operator()} with $n+m, m, n \geq 0$ parameters.\newline

\noindent{}The first $n$ parameters are inputs to the computation.
The remaining $m$ parameters are parameters to the computation.
The types of all parameters are determined by template parameters.

%\begin{example}
\noindent{}The following example provides a partial specialization of an \texttt{equal\_to\_functor} which is a $2,1$ functor.
The functor determines wether two \texttt{std::array} values are are equal by testing wether the corresponding elements
are equal. The first $2$ parameters of the functor determine the inputs to the computation (2-dimensional vectors) and
the last parameter determines how the elements are compared.
\begin{verbatim}
// Base of 2,1 functor.
template<typename LEFT_OPERAND, typename RIGHT_OPERAND, typename EQUAL_TO, typename ENABLED = void>
struct equal_to_functor;

// 2,1 functor
template<size_t LEFT_ELEMENT, size_t LEFT_NUMBER_OF_ELEMENTS,
         size_t RIGHT_ELEMENT, size_t RIGHT_NUMBER_OF_ELEMENTS,
         typename EQUAL_TO>
struct equal_to_functor<std::array<LEFT_ELEMENT, LEFT_NUMBER_OF_ELEMENTS>,
                        std::array<RIGHT_ELEMENT, RIGHT_NUMBER_OF_ELEMENTS>,
                        EQUAL_TO,
                        std::enable_if_t<std::is_same_v<LEFT_ELEMENT, RIGHT_ELEMENT> &&
                                         LEFT_NUMBER_OF_ELEMENTS == 2                &&
                                         RIGHT_NUMBER_OF_ELEMENTS == 2>>
{
   using left_operand_type = std::array<LEFT_ELEMENT, LEFT_DIMENSIONALITY>;
   using right_operand_type = std::array<RIGHT_ELEMENT, RIGHT_DIMENSIONALITY>;
   using equal_to_type = EQUAL_TO;
   auto operator(const left_operand_type& u, const right_operand_type& v, equal_to_type equal_to) const
   {
     return equal_to(u[0], v[0])
         && equal_to(u[1], v[1]);
   }
};
\end{verbatim}
%\end{example}

\noindent{}The provided \texttt{operator()} shall be qualified as \texttt{noexcept} and \texttt{constexpr} if possible.\newline

\noindent{}$n,m$-functors are frequently (partial) specializations of a base functor.
The base functor shall provide a \texttt{ENABLED} template parameter with a default value \texttt{void} which used to perform SFINAE.
\noindent{}The return value of the \texttt{operator()} represents                the operation
$f : V_1 \times \ldots \times V_n \times P_1 \times \ldots \times P_m \rightarrow(partial) R$.
where $V_i$ is the type of the $i$-th input and $P_j$ the type of the $j$-th parameter. $R$ is
the result of the computation.

\subsection{\textit{concept $m$-BinaryFunctor}}
An \textit{$m$-BinaryFunctor} is a \textit{$2,m$-Ary-Functor}.

\subsection{\textit{concept $m$-UnaryFunctor}}
An \textit{$m$-UnaryFunctor} is a \textit{$1,m$-Ary-Functor}.

\subsection{\textit{concept $m$-NullaryFunctor}}
An \textit{$m$-NullaryFunctor} is a \textit{$0,m$-Ary-Functor}.


\section{\texttt{struct vector\_traits}}
\begin{verbatim}
template<typename ELEMENT_TYPE, typename NUMBER_OF_ELEMENTS>
struct vector_traits;
\end{verbatim}

\section{\texttt{struct vector}}
\begin{verbatim}
template<typename TRAITS, typename ENABLED = void>
struct vector;
\end{verbatim}

\subsection{Members}

\subsection{Non Members}
\subsubsection{\texttt{struct dot\_product\_functor}}
The base of partial specializations of this functor is given by
\begin{verbatim}
template<typename LEFT_OPERAND, typename RIGHT_OPERAND, typename ENABLED = void>
struct dot_product_functor;
\end{verbatim}
A \textit{$0$-BinaryFunctor} representing the dot product $\texttt{left\_operand} \cdot \texttt{right\_operand}$.\newline

\noindent{}This libray provides for all \texttt{vector} specializations provided by this library
where \texttt{LEFT\_OPERAND} and \texttt{RIGHT\_OPERAND} are of the same type \texttt{vector\textlangle TRAITS \textrangle}.
A possible implementation is
\begin{verbatim}
template<typename LEFT_OPERAND_TRAITS, typename RIGHT_OPERAND_TRAITS>
struct dot_product_functor<vector<LEFT_OPERAND_TRAITS>,
                           vector<RIGHT_OPERAND_TRAITS>,
                           std::enable_if_t</* implementation */>>
{
  using left_operand_type = vector<LEFT_OPERAND_TRAITS>;
  using right_operand_type = vector<RIGHT_OPERAND_TRAITS>;
  auto operator()(const left_operand_type& u, const right_operand_type& v) const
  { /* implementation */ }
};
\end{verbatim}


\subsubsection{\texttt{struct cross\_product\_functor}}
\begin{verbatim}
template<typename LEFT_OPERAND, typename RIGHT_OPERAND, typename ENABLED = void>
struct cross_product_functor;
\end{verbatim}
A \textit{$0$-BinaryFunctor} representing the cross product $\texttt{left\_operand} \times \texttt{right\_operand}$.\newline

\noindent{}This libray provides specializations for all \texttt{vector} specializations provided by this library
where \texttt{LEFT\_OPERAND} and \texttt{RIGHT\_OPERAND} are of the same type \texttt{vector\textlangle TRAITS \textrangle}
and \texttt{TRAITS::dimensionality} is $3$. A possible implementation is
\begin{verbatim}
template<typename LEFT_OPERAND_TRAITS, typename RIGHT_OPERAND_TRAITS>
struct cross_product_functor<vector<LEFT_OPERAND_TRAITS>,
                             vector<RIGHT_OPERAND_TRAITS>,
                             std::enable_if_t</* implementation */>>
{
  using left_operand_type = vector<LEFT_OPERAND_TRAITS>;
  using right_operand_type = vector<RIGHT_OPERAND_TRAITS>;
  auto operator()(const left_operand_type& u, const right_operand_type& v) const
  { /* implementation */ }
};
\end{verbatim}

\subsubsection{\texttt{square\_euclidean\_norm\_functor}}
An \textit{$0$-UnaryFunctor} representing the square of the Euclidean norm $\left\|\texttt{operand}\right\|^2_2$.\newline

\noindent{}This libray provides specializations for all \texttt{vector} specializations provided by this library.
A possible implementation is
\begin{verbatim}
template<typename OPERAND_TRAITS>
struct squared_euclidean_norm_functor<vector<OPERAND_TRAITS>,
                                      std::enable_if_t</* implementation */>>
{
  using operand_type = vector<OPERAND_TRAITS>;
  auto operator()(const operand_type& v) const
  { /* implementation */ }
};
\end{verbatim}

\subsubsection{\texttt{euclidean\_norm\_functor}}
An \textit{$0$-UnaryFunctor} representing the Euclidean norm $\left\|\texttt{operand}\right\|_2$.\newline

\noindent{}This libray provides specializations for all \texttt{vector} specializations provided by this library.
A possible implementation is
\begin{verbatim}
template<typename OPERAND_TRAITS>
struct euclidean_norm_functor<vector<OPERAND_TRAITS>,
                              std::enable_if_t</* implementation */>>
{
  using operand_type = vector<OPERAND_TRAITS>;
  auto operator()(const operand_type& v) const
  { /* implementation */ }
};
\end{verbatim}

\subsubsection{\texttt{normalize\_functor}}
An \textit{$0$-UnaryFunctor} representing the normalization $\frac{\texttt{operand}}{\left\|\texttt{operand}\right\|_2}$.\newline

\noindent{}This libray provides specializations for all \texttt{vector} specializations provided by this library.
A possible implementation is
\begin{verbatim}
template<typename OPERAND_TRAITS>
struct normalize_functor<vector<OPERAND_TRAITS>,
                         std::enable_if_t</* implementation */>>
{
  using operand_type = vector<OPERAND_TRAITS>;
  auto operator()(const operand_type& v) const
  { /* implementation */ }
};
\end{verbatim}

\subsubsection{\texttt{are\_orthogonal\_functor}}
An \textit{$1$-BinaryFunctor} computing wether two vectors are orthogonal.\newline
\begin{verbatim}
template<typename LEFT_OPERAND, typename RIGHT_OPERAND, typename EQUAL_TO, typename ENABLED = void>
struct are_orthogonal_functor;
\end{verbatim}

\noindent{}This libray provides specializations for all \texttt{vector} specializations provided by this library.
A possible implementation is
\begin{verbatim}
template<typename LEFT_OPERAND_TRAITS,
         typename RIGHT_OPERAND_TRAITS,
         typename EQUAL_TO>
struct are_orthogonal_functor<vector<LEFT_OPERAND_TRAITS>,
                              vector<RIGHT_OPERAND_TRAITS>,
                              EQUAL_TO,
                              std::enable_if_t</* implementation */>>
{
  using left_operand_type = vector<LEFT_OPERAND_TRAITS>;
  using right_operand_type = vector<RIGHT_OPERAND_TRAITS>;
  using equal_to_type = EQUAL_TO;
  bool operator()(const left_operand_type& u, const right_operand_type& v, equal_to_type equal_to) const
  { /* implementation */ }
};
\end{verbatim}

\end{document}
