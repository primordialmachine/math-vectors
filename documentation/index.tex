%%%%%%%%%%%%%%%%%%%%%%%%%%%%%%%%%%%%%%%%%%%%%%%%%%%%%%%%%%%%%%%%%%%%%%%%%%%%%%%%%%%%%%%%%%%%%%%%%%%
%
% Primordial Machine's Vectors Library
% Copyright (C) 2017-2019 Michael Heilmann
%
% This software is provided 'as-is', without any express or implied warranty.
% In no event will the authors be held liable for any damages arising from the
% use of this software.
%
% Permission is granted to anyone to use this software for any purpose,
% including commercial applications, and to alter it and redistribute it
% freely, subject to the following restrictions:
%
% 1. The origin of this software must not be misrepresented;
%    you must not claim that you wrote the original software.
%    If you use this software in a product, an acknowledgment
%    in the product documentation would be appreciated but is not required.
%
% 2. Altered source versions must be plainly marked as such,
%    and must not be misrepresented as being the original software.
%
% 3. This notice may not be removed or altered from any source distribution.
%
%%%%%%%%%%%%%%%%%%%%%%%%%%%%%%%%%%%%%%%%%%%%%%%%%%%%%%%%%%%%%%%%%%%%%%%%%%%%%%%%%%%%%%%%%%%%%%%%%%%

\documentclass[oneside]{book}

\input{common}
\SetOrganization{Primordial Machine}
\SetLibraryName{Math Vectors}
\SetLibraryVersion{1.1}
\SetLibraryRepository{https://github.com/primordialmachine/math-vectors}
\SetAuthor{Michael Heilmann}
\SetEmail{michaelheilmann@primordialmachine.com}


\SetLibraryIncludeFileName{include.hpp}
\SetLibraryIncludesDirectoryPath{primordialmachine/math-vectors/\newline\$(PlatformTarget.toLower())/\$(Configuration.toLower())/includes}

\SetLibraryIncludeDirectiveFilePath{primordialmachine/math/vectors/include.hpp}

\SetLibraryStaticLibrariesDirectoryPath{primordialmachine/math-vectors/\newline\$(PlatformTarget.toLower())/\$(Configuration.toLower())/libraries}
\SetLibraryStaticLibraryFileName{math-vectors.lib}

\SetDocumentType{User Manual}

\begin{document}

\frontmatter

\begin{titlepage}
\maketitle
\end{titlepage}

\tableofcontents
\addtocontents{toc}{\protect\thispagestyle{empty}}
\pagenumbering{gobble}

\mainmatter

\chapter{Synopsis}
C++ 17 library providing floating point vectors over $n$ dimensions.
The library is made available publicly on
\href{\GetLibraryRepository}{Github}
under the
\href{\GetLibraryRepository/blob/master/LICENSE}{MIT License}.

\chapter{Limitations and Restrictions}
The library officially only supports Visual Studio 2017 and Windows 10.

\chapter{Introductory example}
\textit{\color{orange}This library does not provide any examples yet.}
%Examples are located in the \href{\GetLibraryRepository/blob/master/examples}{examples} directory.

% Copyright (c) 2018 Michael Heilmann. All rights reserved.
\chapter{Building under Visual Studio 2017}
\begin{enumerate}
\item Open the solution \texttt{solution.sln} in Microsoft Visual Studio 2017.
\item Batch build everything.
\item The folder \texttt{packages} contains the distribution of the library i.e. include files and the
      static libraries for
  \begin{enumerate}
    \item the platform targets \texttt{x86} and \texttt{x64} and
    \item configurations \texttt{Release} and \texttt{Debug}.
  \end{enumerate}
\item Copy the contents of the \verb+packages+ folder into a directory. Let
      \verb+[library home]+ be a placeholder denoting the path by which that folder
      can be referenced from your project.
\item Add
  \begin{enumerate}
    \item the include path
\texttt{[library home]/\GetLibraryIncludesDirectoryPath}
	and
    \item the library path
\texttt{[library home]/\GetLibraryStaticLibrariesDirectoryPath}
    to your project.
\end{enumerate}
\item Link your project with the library \texttt{\GetLibraryStaticLibraryFileName}.
\item Add the include directive \texttt{\#include "{}\GetLibraryIncludeDirectiveFilePath"{}} where appropriate.
\item You can now use the functionality provided by the library.
\end{enumerate}


\chapter{Library Interface Documentation}

\section{\texttt{namespace primordialmachine}}
The namespace this library is adding its declarations/definitions to.
The added namespace elements are documented below.

%%%%%%%%%%%%%%%%%%%%%%%%%%%%%%%%%%%%%%%%%%%%%%%%%%%%%%%%%%%%%%%%%%%%%%%%%%%%%%%%%%%%%%%%%%%%%%%%%%%
%
% Primordial Machine's Matrices Library
% Copyright (C) 2017-2019 Michael Heilmann
%
% This software is provided 'as-is', without any express or implied warranty.
% In no event will the authors be held liable for any damages arising from the
% use of this software.
%
% Permission is granted to anyone to use this software for any purpose,
% including commercial applications, and to alter it and redistribute it
% freely, subject to the following restrictions:
%
% 1. The origin of this software must not be misrepresented;
%    you must not claim that you wrote the original software.
%    If you use this software in a product, an acknowledgment
%    in the product documentation would be appreciated but is not required.
%
% 2. Altered source versions must be plainly marked as such,
%    and must not be misrepresented as being the original software.
%
% 3. This notice may not be removed or altered from any source distribution.
%
%%%%%%%%%%%%%%%%%%%%%%%%%%%%%%%%%%%%%%%%%%%%%%%%%%%%%%%%%%%%%%%%%%%%%%%%%%%%%%%%%%%%%%%%%%%%%%%%%%%

%%%%%%%%%%%%%%%%%%%%%%%%%%%%%%%%%%%%%%%%%%%%%%%%%%%%%%%%%%%%%%%%%%%%%%%%%%%%%%%%%%%%%%%%%%%%%%%%%%%
\section{\textit{concept Functor}}
A functor is a struct type providing a constant  \texttt{operator()}.
That operator shall be qualified as \texttt{noexcept}     if possible
and shall be qualified as \texttt{constexpr} if possible.  The return
type of \texttt{operator()} shall be of type \texttt{result}    (or a
cv-qualified variant of that).\newline\noindent{}The functor    shall
provide the member type definition \texttt{result\_type} denoting the
type \texttt{result}.

%%%%%%%%%%%%%%%%%%%%%%%%%%%%%%%%%%%%%%%%%%%%%%%%%%%%%%%%%%%%%%%%%%%%%%%%%%%%%%%%%%%%%%%%%%%%%%%%%%%
\section{\textit{concept BinaryFunctor}}
A \textit{BinaryFunctor} is a \textit{Functor}.
Its \texttt{operator()} has two parameters \texttt{left\_operand} of type
\texttt{left\_operand\_type} (or a cv-qualified variant of that)      and 
\texttt{right\_operand} of type \texttt{right\_operand\_type} (or       a
cv-qualified variant of that).\newline

\noindent{}The functor shall provide the member type              definitions
\texttt{left\_operand\_type}  denoting the type \texttt{left\_operand}    and
\texttt{right\_operand\_type} denoting the type      \texttt{right\_operand}.
\newline

\section{\textit{concept BinaryFunctorBase}}
A  \textit{BinaryFunctorBase} is a template struct type of   name \textit{name}
with three template parameters \texttt{LEFT\_OPERAND}, \texttt{RIGHT\_OPERAND},
and \texttt{ENABLED}. The default value of \texttt{ENALBED} is   \texttt{void}.
Specializations of this template may use \texttt{ENABLED} to    perform SFINAE.
A possible implementation is
\texttt{\newline
\noindent{}template\textlangle typename LEFT\_OPERAND, typename RIGHT\_OPERAND,
typename ENABLED = void\textrangle\newline\noindent{}struct \textit{name};    }
\newline

\noindent{}Its partial specializations are \textit{BinaryFunctors}.
%%%%%%%%%%%%%%%%%%%%%%%%%%%%%%%%%%%%%%%%%%%%%%%%%%%%%%%%%%%%%%%%%%%%%%%%%%%%%%%%%%%%%%%%%%%%%%%%%%%
\section{\textit{concept UnaryFunctor}}
An \textit{UnaryFunctor} is \textit{Functor}.
Its \texttt{operator()} has one parameter \texttt{operand}  of type
\texttt{operand\_type} (or a cv-qualified variant of that).\newline

\noindent{}The functor shall provide the member type definition
\texttt{operand\_type} denoting the type      \texttt{operand}.

\section{\textit{concept UnaryFunctorBase}}
An \textit{UnaryFunctorBase} is a template struct type of     name \textit{name}
with two  template parameters \texttt{OPERAND} and \texttt{ENABLED}. The default
value of \texttt{ENALBED} is \texttt{void}. Specializations of this template may
use \texttt{ENABLED} to perform SFINAE.
A possible implementation is
\texttt{\newline
\noindent{}template\textlangle typename OPERAND,
typename ENABLED = void\textrangle\newline\noindent{}struct \textit{name};    }
\newline

\noindent{}Its partial specializations are \textit{UnaryFunctors}.


\section{\texttt{vector\_traits} (struct)}
\begin{verbatim}
template<typename ELEMENT_TYPE, typename NUMBER_OF_ELEMENTS>
struct vector_traits;
\end{verbatim}

\section{\texttt{is\_vector} (struct)}
\texttt{
template \textlangle typename TYPE, typename ENABLED = void\textrangle\newline
struct is\_vector {\newline
  static bool const value = false;\newline
};}

\noindent{}This template defines a compile-time boolean constant \texttt{value}
which can be used to determine whether a type \texttt{TYPE} is a valid vector type.
This library provides specialization for its \texttt{vector} template specializations.

\section{\texttt{dot\_product\_functor} (struct)}
A \textit{BinaryFunctorBase} computing the dot product of two vectors.
A possible implementation is given by
\begin{verbatim}
template<typename LEFT_OPERAND, typename RIGHT_OPERAND, typename ENABLED = void>
struct dot_product_functor;
\end{verbatim}

\section{\texttt{cross\_product\_functor} (struct)}
A \textit{BinaryFunctorBase} computing the cross product of two vectors.
A possible implementation is given by
\begin{verbatim}
template<typename LEFT_OPERAND, typename RIGHT_OPERAND, typename ENABLED = void>
struct cross_product_functor;
\end{verbatim}

\section{\texttt{squared\_euclidean\_norm\_functor} (struct)}
A \textit{UnaryFunctorBase} computing the squared Euclidean norm of a vector.
A possible implementation is given by
\begin{verbatim}
template<typename OPERAND, typename ENABLED = void>
struct squared_euclidean_norm_functor;
\end{verbatim}

\section{\texttt{euclidean\_norm\_functor} (struct)}
An \textit{UnaryFunctorBase} computing the Euclidean norm of a vector.
A possible implementation is given by
\begin{verbatim}
template<typename OPERAND, typename ENABLED = void>
struct euclidean_norm_functor;
\end{verbatim}

%%%%%%%%%%%%%%%%%%%%%%%%%%%%%%%%%%%%%%%%%%%%%%%%%%%%%%%%%%%%%%%%%%%%%%%%%%%%%%%%%%%%%%%%%%%%%%%%%%%
\section{\texttt{vector} (struct)}
\begin{verbatim}
template<typename TRAITS, typename ENABLED = void>
struct vector;
\end{verbatim}
\noindent{}This library provides specializations where \texttt{TRAITS} is a \texttt{vector\_traits}
specialization with the following properties:
\begin{enumerate}
	\item \texttt{ELEMENT\_TYPE} is a type \texttt{TYPE} for which \texttt{is\_scalar\textlangle TYPE \textrangle::value} is \texttt{true}.
	\item \texttt{NUMBER\_OF\_ELEMENTS} is any number between 0 and \texttt{std::numeric\_limits\textlangle size\_t\textrangle\::max()}.
\end{enumerate}

\subsection{Members}

\subsection{Non Members}
\NewDocumentCommand\ForAllVectors{}{for all types \texttt{V}
with
$\texttt{is\_vector\_v\textlangle V\textrangle} = \texttt{true}$}


\NewDocumentCommand\ForAllVectorsAndScalars{}{for all types \texttt{V} and \texttt{S}
with
$\texttt{is\_vector\_v\textlangle V\textrangle} = \texttt{true}$
and
$\texttt{S} = \texttt{element\_type\_t\textlangle V\textrangle}$}


%%%%%%%%%%%%%%%%%%%%%%%%%%%%%%%%%%%%%%%%%%%%%%%%%%%%%%%%%%%%%%%%%%%%%%%%%%%%%%%%%%%%%%%%%%%%%%%%%%%
\subsubsection{\texttt{operator+} (binary, vector vector)}
The arithmetic sum $\vec{u} + \vec{v}$ of two vectors $\vec{u}=\left(u_0,u_1,\ldots,u_{n-1}\right)$
and $\vec{v}=\left(v_0,v_1,\ldots,v_{n-1}\right)$  is defined as $\vec{u} + \vec{v} = \left(u_0   +
v_0, u_1 + v_1, \ldots, u_{n-1} + v_{n-1}\right)$.
%
This library provides specializations of \texttt{binary\_plus\_functor} (see \cite{arithmeticfunctors})
for $\texttt{A}=\texttt{V}$, $\texttt{B}=\texttt{V}$, and $\texttt{R} = \texttt{V}$ \ForAllVectors.

\subsubsection{\texttt{operator+} (unary, vector)}
The arithmetic affirmation $+\vec{v}$ of a vector $\vec{v}=\left(v_0,v_1,\ldots,v_{n-1}\right)$
is defined as $+\vec{v} = \left(+v_0,v_1,\ldots,v_{n-1}\right)$.
%
This library provides specializations of \texttt{unary\_plus\_functor} (see \cite{arithmeticfunctors})
for $\texttt{A}=\texttt{V}$ and $\texttt{R} = \texttt{V}$ \ForAllVectors.

\subsubsection{\texttt{operator+=} (vector vector)}
This library provides specializations of \texttt{plus\_assignment\_functor}  (see \cite{arithmeticfunctors})
for $\texttt{A} = \texttt{V}$, $\texttt{B} = \texttt{V}$ \ForAllVectors.

%%%%%%%%%%%%%%%%%%%%%%%%%%%%%%%%%%%%%%%%%%%%%%%%%%%%%%%%%%%%%%%%%%%%%%%%%%%%%%%%%%%%%%%%%%%%%%%%%%%
\subsubsection{\texttt{operator-} (binary, vector vector)}
The arithmetic difference $\vec{u} - \vec{v}$ of two vector $\vec{u}=\left(u_0,u_1,\ldots,u_{n-1}\right)$
and $\vec{v}=\left(v_0,v_1,\ldots,v_{n-1}\right)$ is defined as $\vec{u}+\vec{v} = \left(u_0 - v_0,
u_1 - v_1,\ldots,u_{n-1}+v_{n-1}\right)$.
%
This library provides specializations of \texttt{binary\_minus\_functor} (see \cite{arithmeticfunctors})
for $\texttt{A}=\texttt{V}$, $\texttt{B}=\texttt{V}$, and $\texttt{R} = \texttt{V}$ \ForAllVectors.

\subsubsection{\texttt{operator-} (unary, vector)}
The arithmetic negation $-\vec{v}$ of a vector $\vec{v}=\left(v_0,v_1,\ldots,v_{n-1}\right)$
is defined as $-\vec{v} = \left(-v_0, -v_1,\ldots, -v_{n-1}\right)$.
%
This library provides specializations of \texttt{unary\_minus\_functor} (see \cite{arithmeticfunctors})
for $\texttt{A}=\texttt{V}$ and $\texttt{R} = \texttt{V}$ \ForAllVectors.

\subsubsection{\texttt{operator-=} (vector vector)}
This library provides specializations of \texttt{minus\_assignment\_functor}  (see \cite{arithmeticfunctors})
for $\texttt{A} = \texttt{V}$, $\texttt{B} = \texttt{V}$ \ForAllVectors.

%%%%%%%%%%%%%%%%%%%%%%%%%%%%%%%%%%%%%%%%%%%%%%%%%%%%%%%%%%%%%%%%%%%%%%%%%%%%%%%%%%%%%%%%%%%%%%%%%%%
\subsubsection{\texttt{operator*} (binary, vector vector)}
The component-wise product $\vec{u} \cdot \vec{v}$ of two vector $\vec{u}=\left(u_0, u_1, \ldots, u_{n-1}\right)$
and $\vec{v}=\left(v_0,v_1,\ldots,v_{n-1}\right)$ is defined as $\vec{u} \cdot \vec{v} = \left(u_0 \cdot v_0,
u_1 \cdot v_1,\ldots, u_{n-1} \cdot v_{n-1}\right)$.
%
This library provides specializations of \texttt{binary\_star\_functor} (see \cite{arithmeticfunctors})
for $\texttt{A}=\texttt{V}$, $\texttt{B}=\texttt{V}$, and $\texttt{R} = \texttt{V}$ \ForAllVectors.

\subsubsection{\texttt{operator*=} (binary, vector vector)}
This library provides specializations of \texttt{star\_assignment\_functor}  (see \cite{arithmeticfunctors})
for $\texttt{A} = \texttt{V}$, $\texttt{B} = \texttt{V}$ \ForAllVectors.

\subsubsection{\texttt{operator*} (binary, vector scalar)}
The vector scalar product $\vec{v} \cdot s$ of a vector $\vec{v}=\left(v_0, v_1, \ldots, v_{n-1}\right)$
and a scalar $s$ is defined as $\vec{v} \cdot s = \left(v_0 \cdot s, v_1 \cdot s, \ldots, v_{n-1} \cdot
s\right)$.
%
This library provides specializations of \texttt{binary\_star\_functor} (see \cite{arithmeticfunctors})
for $\texttt{A}=\texttt{V}$, $\texttt{B}=\texttt{S}$, and $\texttt{R} = \texttt{T}$ \ForAllVectorsAndScalars.

\subsubsection{\texttt{operator*=} (binary, vector scalar)}
This library provides specializations of \texttt{star\_assignment\_functor}  (see \cite{arithmeticfunctors})
for $\texttt{A} = \texttt{V}$, $\texttt{B} = \texttt{S}$ \ForAllVectorsAndScalars.

\subsubsection{\texttt{operator*} (binary, scalar vector)}
The scalar vector product $s \cdot \vec{v}$ is defined as $s \cdot \vec{v} = \vec{v} \cdot s$.
%
This library provides specializations of \texttt{binary\_star\_functor} (see \cite{arithmeticfunctors})
for $\texttt{A}=\texttt{S}$, $\texttt{B}=\texttt{V}$, and $\texttt{R} = \texttt{V}$ \ForAllVectorsAndScalars.
%for all types \texttt{V} with \texttt{is\_vector\_v\textlangle V\textrangle = true}
%and $\texttt{S} = \texttt{element\_type\_t\textlangle V\textrangle}$.

%%%%%%%%%%%%%%%%%%%%%%%%%%%%%%%%%%%%%%%%%%%%%%%%%%%%%%%%%%%%%%%%%%%%%%%%%%%%%%%%%%%%%%%%%%%%%%%%%%%
\subsubsection{\texttt{operator/} (binary, vector scalar)}
The vector scalar quotient $\vec{v} / s$ of a vector $\vec{v}=\left(v_0, v_1, \ldots, v_{n-1}\right)$
and a scalar $s$ is defined as $\vec{v} \cdot s = \left(v_0 / s, v_1 / s, \ldots, v_{n-1} /
s\right)$.
%
This library provides specializations of \texttt{binary\_slash\_functor} (see \cite{arithmeticfunctors})
for $\texttt{A}=\texttt{V}$, $\texttt{B}=\texttt{S}$, and $\texttt{R} = \texttt{V}$ \ForAllVectorsAndScalars.

\subsubsection{\texttt{operator/=} (binary, vector scalar)}
This library provides specializations of \texttt{star\_assignment\_functor}  (see \cite{arithmeticfunctors})
for $\texttt{A} = \texttt{V}$, $\texttt{B} = \texttt{S}$ \ForAllVectorsAndScalars.

%%%%%%%%%%%%%%%%%%%%%%%%%%%%%%%%%%%%%%%%%%%%%%%%%%%%%%%%%%%%%%%%%%%%%%%%%%%%%%%%%%%%%%%%%%%%%%%%%%%
\subsubsection{\texttt{operator==} (binary, vector vector)}
The equal to relation $\vec{u} = \vec{v}$ of two vector $\vec{u}=\left(u_0,u_1,\ldots,u_{n-1}\right)$
and $\vec{v}=\left(v_0,v_1,\ldots,v_{n-1}\right)$ is defined as $\vec{u} = \vec{v} = u_0 = v_0
\wedge u_1 = v_1 \wedge \ldots \wedge u_{n-1} = v_{n-1}$.
%
This library provides specializations of \texttt{equal\_to\_functor} (see \cite{relationalfunctors})
for $\texttt{A}=\texttt{V}$, $\texttt{B}=\texttt{V}$ \ForAllVectors.

\subsubsection{\texttt{operator!=} (binary, vector vector)}
The not equal to relation $\vec{u} \neq \vec{v}$ of two vector $\vec{u}=\left(u_0,u_1,\ldots,u_{n-1}\right)$
and $\vec{v}=\left(v_0,v_1,\ldots,v_{n-1}\right)$ is defined as $\vec{u} \neq \vec{v} = u_0 \neq v_0
\vee u_1 \neq v_1 \vee \ldots \vee u_{n-1} \neq v_{n-1}$.
%
This library provides specializations of \texttt{not\_equal\_to\_functor} (see \cite{relationalfunctors})
for $\texttt{A}=\texttt{V}$, $\texttt{B}=\texttt{V}$ \ForAllVectors.

%%%%%%%%%%%%%%%%%%%%%%%%%%%%%%%%%%%%%%%%%%%%%%%%%%%%%%%%%%%%%%%%%%%%%%%%%%%%%%%%%%%%%%%%%%%%%%%%%%%
\subsubsection{\texttt{dot\_product\_functor} (struct)}
This library provides \textit{BinaryFunctor} specializations of this \textit{BinaryFunctorBase}
where \texttt{LEFT\_OPERAND} and \texttt{RIGHT\_OPERAND} are specializations of \texttt{vector}
with the same number of elements $N$. The result type \texttt{result\_type}          is\newline
\texttt{std::common\_type\_t\textlangle LEFT\_OPERAND\_TRAITS::element\_type, RIGHT\_OPERAND\_TRAITS::element\_type\textrangle}\newline
 where \texttt{LEFT\_OPERAND\_TRAITS} and \texttt{RIGHT\_OPERAND\_TRAITS} are the traits of these specializations.

%%%%%%%%%%%%%%%%%%%%%%%%%%%%%%%%%%%%%%%%%%%%%%%%%%%%%%%%%%%%%%%%%%%%%%%%%%%%%%%%%%%%%%%%%%%%%%%%%%%
\subsubsection{\texttt{cross\_product\_functor} (struct)}
This library provides \textit{BinaryFunctor} specializations of this \textit{BinaryFunctorBase}
where \texttt{LEFT\_OPERAND} and \texttt{RIGHT\_OPERAND} are specializations of \texttt{vector}
with the same number of elements $3$. The result type \texttt{result\_type}                  is
\texttt{vector\textlangle RESULT\_TRAITS\textrangle} where\newline
\texttt{
RESULT\_TRAITS:=vector\_traits\textlangle std::common\_type\_t\textlangle{}LEFT\_OPERAND\_TRAITS::element\_type,\newline%
\hphantom{RESULT\_TRAITS:=vector\_traits\textlangle std::common\_type\_t\textlangle}RIGHT\_OPERAND\_TRAITS::element\_type\textrangle,\newline%
\hphantom{RESULT\_TRAITS:=vector\_traits\textlangle}3\textrangle}.

%%%%%%%%%%%%%%%%%%%%%%%%%%%%%%%%%%%%%%%%%%%%%%%%%%%%%%%%%%%%%%%%%%%%%%%%%%%%%%%%%%%%%%%%%%%%%%%%%%%
\subsubsection{\texttt{squared\_euclidean\_norm\_functor} (struct)}
This library provides \textit{UnaryFunctor} specializations of this \textit{UnaryFunctorBase}
where \texttt{OPERAND} is specializations of \texttt{vector}. The result type        \texttt{
result\_type} is \texttt{OPERAND\_TRAITS::element\_type\textrangle}.

%%%%%%%%%%%%%%%%%%%%%%%%%%%%%%%%%%%%%%%%%%%%%%%%%%%%%%%%%%%%%%%%%%%%%%%%%%%%%%%%%%%%%%%%%%%%%%%%%%%
\subsubsection{\texttt{euclidean\_norm\_functor} (struct)}
This library provides \textit{UnaryFunctor} specializations of this \textit{UnaryFunctorBase}
where \texttt{OPERAND} is specializations of \texttt{vector}. The result type        \texttt{
result\_type} is \texttt{OPERAND\_TRAITS::element\_type}.

%%%%%%%%%%%%%%%%%%%%%%%%%%%%%%%%%%%%%%%%%%%%%%%%%%%%%%%%%%%%%%%%%%%%%%%%%%%%%%%%%%%%%%%%%%%%%%%%%%%
\subsubsection{\texttt{normalize\_functor} (struct)}
A \textit{UnaryFunctorBase} computing the normalized vector of a vector.
A possible implementation is given by
\begin{verbatim}
template<typename OPERAND, typename NORM, typename ENABLED = void>
struct normalize_functor;
\end{verbatim}
\texttt{NORM} is an additional template argument denoting the type of norm functor.\newline

This library provides \textit{UnaryFunctor} specializations of this \textit{UnaryFunctorBase} where
\texttt{OPERAND} is specializations of \texttt{vector}. The result type \texttt{result\_type}    is
\texttt{vector\textlangle RESULT\_TRAITS\textrangle} where \texttt{RESULT\_TRAITS}               is
\texttt{OPERAND\_TRAITS}.\newline

The \texttt{operator()} has a third parameter  \texttt{norm} which defaults to
\texttt{euclidean\_norm\textlangle vector\textlangle RESULT\_TRAITS\textrangle
\textrangle} through which any other norm can be passed.\newline

\textit{\textcolor{orange}{The \textit{ParameterizedBinaryFunctor(Base)} concepts - like
\texttt{normalize\_functor}       - subsumes the \textit{BinaryFunctor(Base)}  concepts.
Add the \textit{ParameterizedBinaryFunctor(Base) concepts.}}}

%%%%%%%%%%%%%%%%%%%%%%%%%%%%%%%%%%%%%%%%%%%%%%%%%%%%%%%%%%%%%%%%%%%%%%%%%%%%%%%%%%%%%%%%%%%%%%%%%%%
\subsubsection{\texttt{are\_orthogonal\_functor} (struct)}
A \textit{BinaryFunctorBase} computing if two vectors are orthogonal.
A possible implementation is given by
\begin{verbatim}
template<typename OPERAND, typename NORM, typename EQUAL_TO, typename ENABLED = void>
struct are_orthogonal_functor;
\end{verbatim}
\texttt{EQUAL\_TO} is an additional template argument denoting the type of an equal to functor.\newline

This library provides \textit{UnaryFunctor} specializations of this \textit{UnaryFunctorBase}
where \texttt{OPERAND} is specializations of \texttt{vector}. The result type        \texttt{
result\_type} is \texttt{bool}.\newline

The \texttt{operator()} has a third parameter \texttt{norm}   which accepts an
\texttt{equal\_to\_functor\textlangle SCALAR, SCALAR\textrangle}         where
\texttt{SCALAR} is \texttt{std::common\_type\_t                    \textlangle
LEFT\_OPERAND\_TRAITS::element\_type,    RIGHT\_OPERAND\_TRAITS::element\_type
\textrangle}.\newline

\textit{\textcolor{orange}{The \textit{ParameterizedBinaryFunctor(Base)} concepts - like
\texttt{are\_orthogonal\_functor} - subsumes the \textit{BinaryFunctor(Base)}  concepts.
Add the \textit{ParameterizedBinaryFunctor(Base) concepts.}}}

\subsubsection{\texttt{are\_orthogonal} (function)}
A \textit{BinaryFunction} computing if two vector

\input{bibliography}

\end{document}
